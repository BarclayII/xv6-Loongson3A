\documentclass{report}
\usepackage{color}
\usepackage{listings}
\usepackage{amsmath}
\usepackage{amssymb}

\def \hilite#1{\textcolor{red}{#1}}

\lstset{basicstyle=\footnotesize\ttfamily}

\lstdefinestyle{c}{
	basicstyle=\footnotesize\ttfamily\color{black},
	keywordstyle=\color{blue},
	commentstyle=\color{green},
	stringstyle=\color{red},
	numberstyle=\color{red},
	showspaces=false,
	showstringspaces=false,
	language=C
}

\newenvironment{hilight}{\color{red}}{\color{black}}

\begin{document}
	\title{
		Porting xv6 to Loongson 3A, a CPU in MIPS64
		family
	}
	\author{Gan Quan}
	\maketitle
	
	\tableofcontents
	
	\chapter{Operating system interfaces}
	The job of an operating system is to share a computer among multiple programs
	and to provide a more useful set of services than the hardware alone supports. The
	operating system manages and abstracts the low-level hardware, so that, for example, a
	word processor need not concern itself with which type of disk hardware is being
	used. It also multiplexes the hardware, allowing many programs to share the computer
	and run (or appear to run) at the same time. Finally, operating systems provide
	controlled ways for programs to interact, so that they can share data or work together.
	
	An operating system provides services to user programs through an interface.
	Designing a good interface turns out to be difficult. On the one hand, we would like
	the interface to be simple and narrow because that makes it easier to get the
	implementation right. On the other hand, we may be tempted to offer many sophisticated
	features to applications. The trick in resolving this tension is to design interfaces that
	rely on a few mechanisms that can be combined to provide much generality.
	
	This book uses a single operating system as a concrete example to illustrate operating 
	system concepts. That operating system, xv6, provides the basic interfaces introduced 
	by Ken Thompson and Dennis Ritchie's Unix operating system, as well as mimicking
	Unix's internal design. Unix provides a narrow interface whose mechanisms
	combine well, offering a surprising degree of generality. This interface has been so
	successful that modern operating systems--BSD, Linux, Mac OS X, Solaris, and even,
	to a lesser extent, Microsoft Windows--have Unix-like interfaces. Understanding xv6
	is a good start toward understanding any of these systems and many others.
	
	xv6 takes the traditional form of a \emph{kernel}, a special program that provides 
	services to running programs. Each running program, called a
	\emph{process}, has memory containing instructions, data, and a stack. The instructions 
	implement the program's computation. The data are the variables on which the computation 
	acts. The stack organizes the program's procedure calls.
	
	When a process needs to invoke a kernel service, it invokes a procedure call in
	the operating system interface. Such a procedure is called a \emph{system call}. The system
	call enters the kernel; the kernel performs the service and returns. Thus a process
	alternates between executing in \emph{user space} and \emph{kernel space}.
	
	The kernel uses the CPU's hardware protection mechanisms to ensure that each
	process executing in user space can access only its own memory. The kernel executes
	with the hardware privileges required to implement these protections; user programs
	execute without those privileges. When a user program invokes a system call, the
	hardware raises the privilege level and starts executing a pre-arranged function in the
	kernel.
	
	The collection of system calls that a kernel provides is the interface that user pro-
	grams see. The xv6 kernel provides a subset of the services and system calls that Unix
	kernels traditionally offer.
	
	\textcolor{red}{
		The current implementation, dubbed xv6', (is planned to) add some more features and
		functions provided my more modern operating systems. The system calls supported by xv6'
		is listed in \figurename{\ref{syscalls}}.  Currently, only \texttt{write()} is implemented,
		and the \texttt{write()} call can only print contents to a serial.
	}
	
	\begin{figure}
		\centering
		\begin{tabular}{ll}
			System call (with arguments) & Description \\
			\hline
			\texttt{fork()} & Create process \\
			\texttt{exit(\textcolor{red}{code})} & Terminate current process \\
			\textcolor{red}{\texttt{waitpid(pid, stat, opts)}} & Wait for a child process to exit \\
			\texttt{kill(pid, \textcolor{red}{sig})} & Send signal to a process with given PID \\
			\texttt{getpid()} & Get current process ID \\
			\texttt{sleep(n)} & Sleep for \texttt{n} seconds \\
			\textcolor{red}{\texttt{usleep(n)}} & Sleep for (about) \texttt{n} microseconds \\
			\textcolor{red}{\texttt{execve(name, argv, envp)}} & Load and execute a file \\
			\texttt{sbrk(incr)} & Grow or shrink process heap \\
			\texttt{open(name, flags)} & Open a file or device in some mode \\
			\texttt{read(fd, buf, len)} & Read from opened file or device \\
			\texttt{write(fd, buf, len)} & Write to opened file or device \\
			\texttt{close(fd)} & Close an opened file \\
			\texttt{dup(fd)} & Duplicate a file descriptor \\
			\textcolor{red}{\texttt{dup2(old, new)}} & Duplicate a file descriptor with preference \\
			\texttt{pipe(p)} & Create a pipe \\
			\texttt{chdir(dir)} & Change current directory \\
			\texttt{mkdir(dir)} & Make directory \\
			\texttt{mknod(name, major, minor)} & Make a device file \\
			\texttt{fstat(fd, stat)} & Fetch file info \\
			\texttt{link(old, new)} & Create another name for given file \\
			\texttt{unlink(name)} & Remove a name or a file \\
		\end{tabular}
		\caption{List of system calls supported by xv6', with difference from xv6 colored red}
		\label{syscalls}
	\end{figure}
	
	The rest of this chapter outlines xv6's services--processes, memory, file descriptors, 
	pipes, and file system--and illustrates them with code snippets and discussions of
	how the shell uses them. The shell's use of system calls illustrates how carefully they
	have been designed.
	
	The shell is an ordinary program that reads commands from the user and executes them, 
	and is the primary user interface to traditional Unix-like systems. The fact
	that the shell is a user program, not part of the kernel, illustrates the power of the 
	system call interface: there is nothing special about the shell. It also means that the shell
	is easy to replace; as a result, modern Unix systems have a variety of shells to choose
	from, each with its own user interface and scripting features. The xv6 shell is a simple
	implementation of the essence of the Unix Bourne shell.
	\textcolor{red}{
		The shell is not implemented yet.
	}
	
	\section{Processes and memory}
	An xv6 process consists of user-space memory (instructions, data, and stack) and
	per-process state private to the kernel. Xv6 can \emph{time-share} processes: it
	transparently switches the available CPUs among the set of processes waiting to execute. 
	When a process is not executing, xv6 saves its CPU registers, restoring them when it next 
	runs the process. The kernel associates a process identifier, or \emph{PID}, with each process.
	
	A process may create a new process using the \texttt{fork} system call. \texttt{fork} creates a
	new process, called the \emph{child process}, with exactly the same memory contents as the
	calling process, called the \emph{parent process}. \texttt{fork} returns in both the parent and the
	child. In the parent, \texttt{fork} returns the child's pid; in the child, it returns zero. 
	For example, consider the following program fragment:
	
	\begin{lstlisting}[style=c]
	int pid = fork();
	int status;
	if(pid > 0){
		printf("parent: child=%d\n", pid);
		waitpid(pid, &status, 0);
		printf("child %d is done\n", pid);
	} else if(pid == 0){
		printf("child: exiting\n");
		exit(0);
	} else {
		printf("fork error\n");
	}
	\end{lstlisting}
	
	The exit system call causes the calling process to stop executing and to release
	resources such as memory and open files. The \texttt{waitpid} system call returns the pid of an
	exited child of the current process; if none of the caller's children has exited, \texttt{waitpid}
	waits for one to do so. In the example, the output lines
	\begin{lstlisting}
	parent: child=1234
	child: exiting
	\end{lstlisting}
	might come out in either order, depending on whether the parent or child gets to its
	\texttt{printf} call first. After the child exits the parent's wait returns, causing the parent to
	print
	\begin{lstlisting}
	parent: child 1234 is done
	\end{lstlisting}
	Note that the parent and child were executing with different memory and different
	registers: changing a variable in one does not affect the other.
	
	The \texttt{execve} system call replaces the calling process's memory with a new memory
	image loaded from a file stored in the file system. The file must have a particular format, 
	which specifies which part of the file holds instructions, which part is data, at
	which instruction to start, etc. xv6 uses the ELF format, which Chapter 3 discusses in
	more detail. When \texttt{execve} succeeds, it does not return to the calling program; instead,
	the instructions loaded from the file start executing at the entry point declared in the
	ELF header. \texttt{execve} takes two arguments: the name of the file containing the executable
	and an array of string arguments. For example:
	
	\begin{lstlisting}[style=c]
	char *argv[3];
	argv[0] = "echo";
	argv[1] = "hello";
	argv[2] = 0;
	execve("/bin/echo", argv, NULL);
	printf("exec error\n");
	\end{lstlisting}
	
	This fragment replaces the calling program with an instance of the program
	\texttt{/bin/echo} running with the argument list \texttt{echo hello}. Most 
	programs ignore the first argument, which is conventionally the name of the program.
	
	The xv6 shell uses the above calls to run programs on behalf of users. The main
	structure of the shell is simple. The main loop reads the input on the
	command line using \texttt{getcmd}. Then it calls \texttt{fork}, which creates a 
	copy of the shell process. The parent shell calls \texttt{waitpid}, while the child process 
	runs the command. For 
	example, if the user had typed "echo hello" at the prompt, \texttt{runcmd} would have been
	called with \texttt{"echo hello"} as the argument. \texttt{runcmd} runs the actual command.
	For \texttt{"echo hello"}, it would call \texttt{execve}. If \texttt{execve} succeeds 
	then the child will execute 
	instructions from \texttt{echo} instead of \texttt{runcmd}. At some point \texttt{echo} will call 
	\texttt{exit},
	which will cause the parent to return from \texttt{waitpid} in main. You might wonder
	why \texttt{fork} and \texttt{execve} are not combined in a single call; we will see later that separate
	calls for creating a process and loading a program is a clever design.
	
	Xv6 allocates most user-space memory implicitly: \texttt{fork} allocates the memory 
	required for the child's copy of the parent's memory, and \texttt{execve} allocates enough memory
	to hold the executable file. A process that needs more memory at run-time (perhaps
	for \texttt{malloc}) can call \texttt{sbrk(n)} to grow its data memory by n bytes; 
	\texttt{sbrk} returns the location of the new memory.
	
	Xv6 does not provide a notion of users or of protecting one user from another; in
	Unix terms, all xv6 processes run as root. \textcolor{red}{
		If there's still time, perhaps \emph{user access control} could be added into xv6'?
	}
	
	\section{I/O and File descriptors}
	A \emph{file descriptor} is a small integer representing a kernel-managed object that
	a process may read from or write to. A process may obtain a file descriptor by opening 
	a file, directory, or device, or by creating a pipe, or by duplicating an existing 
	descriptor. For simplicity we'll often refer to the object a file descriptor refers to as a
	"file"; the file descriptor interface abstracts away the differences between files, pipes,
	and devices, making them all look like streams of bytes.
	
	Internally, the xv6 kernel uses the file descriptor as an index into a per-process table, 
	so that every process has a private space of file descriptors starting at zero. By
	convention, a process reads from file descriptor 0 (standard input, \texttt{STDIN\_FILENO}), 
	writes output to file descriptor 1 (standard output, \texttt{STDOUT\_FILENO}), 
	and writes error messages to file descriptor 2 (standard
	error, \texttt{STDERR\_FILENO}). As we will see, the shell exploits the convention to 
	implement I/O redirection
	and pipelines. The shell ensures that it always has three file descriptors open,
	which are by default file descriptors for the console.
	
	The \texttt{read} and \texttt{write} system calls read bytes from and write bytes to open files
	named by file descriptors. The call \texttt{read(fd, buf, n)} reads at most \texttt{n} bytes from the
	file descriptor \texttt{fd}, copies them into \texttt{buf}, and returns the number of bytes read. Each
	file descriptor that refers to a file has an offset associated with it. Read reads data
	from the current file offset and then advances that offset by the number of bytes read:
	a subsequent \texttt{read} will return the bytes following the ones returned by the first read.
	When there are no more bytes to read, \texttt{read} returns zero to signal the end of the file. 
	
	The call \texttt{write(fd, buf, n)} writes \texttt{n} bytes from \texttt{buf} to the file 
	descriptor \texttt{fd} and
	returns the number of bytes written. Fewer than \texttt{n} bytes are written only when an 
	error occurs. Like \texttt{read}, \texttt{write} writes data at the current file offset and then advances
	that offset by the number of bytes written: each \texttt{write} picks up where the previous
	one left off.
	
	The following program fragment (which forms the essence of \texttt{cat}) copies data
	from its standard input to its standard output. If an error occurs, it writes a message
	to the standard error.
	
	\begin{lstlisting}[style=c]
	char buf[512];
	int n;
	for(;;){
		n = read(STDIN_FILENO, buf, sizeof buf);
		if(n == 0)
			break;
		if(n < 0){
			fprintf(STDERR_FILENO, "read error\n");
			exit(1);
		}
		if(write(STDOUT_FILENO, buf, n) != n){
			fprintf(STDERR_FILENO, "write error\n");
			exit(1);
		}
	}
	\end{lstlisting}
	
	The important thing to note in the code fragment is that \texttt{cat} doesn't know whether it
	is reading from a file, console, or a pipe. Similarly \texttt{cat} doesn't know whether it is
	printing to a console, a file, or whatever. The use of file descriptors and the convention that 
	file descriptor 0 is input and file descriptor 1 is output allows a simple implementation of 
	\texttt{cat}.
	
	The \texttt{close} system call releases a file descriptor, making it free for reuse by a future 
	\texttt{open}, \texttt{pipe}, or \texttt{dup} system call (see below). A newly allocated file 
	descriptor is always the lowest-numbered unused descriptor of the current process.
	
	File descriptors and \texttt{fork} interact to make I/O redirection easy to implement.
	\texttt{fork} copies the parent's file descriptor table along with its memory, so that the child
	starts with exactly the same open files as the parent. The system call \texttt{execve} replaces the
	calling process's memory but preserves its file table. This behavior allows the shell to
	implement I/O redirection by forking, reopening chosen file descriptors, and then \texttt{execve}-ing the 
	new program. Here is a simplified version of the code a shell runs for the
	command \texttt{cat <input.txt}:
	
	\begin{lstlisting}[style=c]
	char *argv[2];
	argv[0] = "cat";
	argv[1] = NULL;
	if(fork() == 0) {
		fd = open("input.txt", O_RDONLY);
		dup2(fd, STDIN_FILENO);
		close(fd);
		execve("cat", argv, NULL);
	}
	\end{lstlisting}
	
	\textcolor{red}{
		Unlike original xv6, child process in this code snippet opens \texttt{input.txt} in a new file
		descriptor, which is copied to the standard input file descriptor (see below), closing the standard input
		during the process (as specified by \texttt{dup2(2)} in POSIX standard).  The child then invokes
		\texttt{execve}, with the newly-allotted file descriptor closed, but the (duplicated) standard 
		input preserved.  This snippet somehow resembles more to modern programs.
	}
	
	The code for I/O redirection in the xv6 shell works in exactly this way. Recall 
	that at this point in the code the shell has already forked the child shell and that
	\texttt{runcmd} will call \texttt{execve} to load the new program. Now it should be clear why it is a
	good idea that \texttt{fork} and \texttt{execve} are separate calls. This separation allows the shell
	to fix up the child process before the child runs the intended program.
	
	Although fork copies the file descriptor table, each underlying file offset is shared
	between parent and child. Consider this example:
	
	\begin{lstlisting}[style=c]
	if((pid = fork()) == 0) {
		write(STDOUT_FILENO, "hello ", 6);
		exit(0);
	} else {
		waitpid(pid, &status, 0);
		write(STDOUT_FILENO, "world\n", 6);
	}
	\end{lstlisting}
	
	At the end of this fragment, the file attached to file descriptor 1 will contain the data
	hello world. The \texttt{write} in the parent (which, thanks to \texttt{waitpid}, runs only after the
	child is done) picks up where the child's \texttt{write} left off. This behavior helps produce
	sequential output from sequences of shell commands, like \texttt{(echo hello; echo world)
	>output.txt}.
	
	The \texttt{dup} \textcolor{red}{(and \texttt{dup2} in xv6')} system call duplicates an existing file 
	descriptor, returning a new one that
	refers to the same underlying I/O object. Both file descriptors share an offset, just as
	the file descriptors duplicated by \texttt{fork} do. This is another way to write \texttt{hello world}
	into a file:
	
	\begin{lstlisting}[style=c]
	fd = dup(STDOUT_FILENO);
	write(STDOUT_FILENO, "hello ", 6);
	write(fd, "world\n", 6);
	\end{lstlisting}
	
	Two file descriptors share an offset if they were derived from the same original
	file descriptor by a sequence of \texttt{fork} and \texttt{dup} calls. Otherwise file descriptors do not
	share offsets, even if they resulted from open calls for the same file. \texttt{dup} allows shells
	to implement commands like this: \texttt{ls existing-file non-existing-file > tmp1
	2>\&1}. The \texttt{2>\&}1 tells the shell to give the command a file descriptor 2 that is a duplicate 
	of descriptor 1. Both the name of the existing file and the error message for the
	non-existing file will show up in the file \texttt{tmp1}. The xv6 shell doesn't support I/O 
	redirection for the error file descriptor, but now you know how to implement it.
	\textcolor{red}{
		Standard error redirection would be implemented in xv6'.
	}
	
	File descriptors are a powerful abstraction, because they hide the details of what
	they are connected to: a process writing to file descriptor 1 may be writing to a file, to
	a device like the console, or to a pipe.
	
	\section{Pipes}
	A \emph{pipe} is a small kernel buffer exposed to processes as a pair of file descriptors,
	one for reading and one for writing. Writing data to one end of the pipe makes that
	data available for reading from the other end of the pipe. Pipes provide a way for
	processes to communicate, \textcolor{red}{
		that is, it is a form of \emph{inter-process communication (IPC)}.  Other forms of IPC
		include \emph{shared memory} and \emph{signals}, which would be implemented (potentially
		much) later in xv6'.
	}
	
	The following example code runs the program \texttt{wc} with standard input connected
	to the read end of a pipe.
	
	\begin{lstlisting}[style=c]
	int p[2];
	char *argv[2];
	argv[0] = "wc";
	argv[1] = NULL;
	pipe(p);
	if(fork() == 0) {
		dup2(p[0], STDIN_FILENO);
		close(p[0]);
		close(p[1]);
		execve("/bin/wc", argv, NULL);
	} else {
		write(p[1], "hello world\n", 12);
		close(p[0]);
		close(p[1]);
	}
	\end{lstlisting}
	
	The program calls \texttt{pipe}, which creates a new pipe and records the read and write file
	descriptors in the array \texttt{p}. After \texttt{fork}, both parent and child have file 
	descriptors referring to the pipe. The child \texttt{dup}s the read end onto file descriptor 0, 
	closes the file descriptors in \texttt{p}, and \texttt{execve}s \texttt{wc}. When \texttt{wc} reads 
	from its standard input, it reads from the
	pipe. The parent writes to the write end of the pipe and then closes both of its file
	descriptors.
	
	If no data is available, a \texttt{read} on a pipe waits for either data to be written or all
	file descriptors referring to the write end to be closed; in the latter case, read will 
	return 0, just as if the end of a data file had been reached. The fact that \texttt{read} blocks
	until it is impossible for new data to arrive is one reason that it's important for the
	child to close the write end of the pipe before executing \texttt{wc} above: if one of \texttt{wc}'s file
	descriptors referred to the write end of the pipe, wc would never see end-of-file.
	
	The xv6 shell implements pipelines such as \texttt{grep fork sh.c | wc -l} in a 
	manner similar to the above code. The child process creates a pipe to connect the
	left end of the pipeline with the right end. Then it calls \texttt{runcmd} for the left end of the
	pipeline and \texttt{runcmd} for the right end, and waits for the left and the right ends to 
	finish, by calling \texttt{waitpid} twice. The right end of the pipeline may be a command that itself
	includes a pipe (e.g., \texttt{a | b | c}), which itself forks two new child processes (one for \texttt{b}
	and one for \texttt{c}). Thus, the shell may create a tree of processes. The leaves of this tree
	are commands and the interior nodes are processes that wait until the left and right
	children complete. In principle, you could have the interior nodes run the left end of
	a pipeline, but doing so correctly would complicate the implementation.
	
	Pipes may seem no more powerful than temporary files: the pipeline
	\begin{lstlisting}
	echo hello world | wc
	\end{lstlisting}
	could be implemented without pipes as
	\begin{lstlisting}
	echo hello world >/tmp/xyz; wc </tmp/xyz
	\end{lstlisting}
	
	There are at least three key differences between pipes and temporary files. First, pipes
	automatically clean themselves up; with the file redirection, a shell would have to be
	careful to remove \texttt{/tmp/xyz} when done. Second, pipes can pass arbitrarily long
	streams of data, while file redirection requires enough free space on disk to store all
	the data. Third, pipes allow for synchronization: two processes can use a pair of pipes
	to send messages back and forth to each other, with each read blocking its calling pro-
	cess until the other process has sent data with \texttt{write}.
	
	\section{File system}
	\textcolor{red}{
		As long as my box doesn't have a hard disk or other external storage, I have to implement
		a file system like \texttt{tmpfs}.  Nevertheless, a file system should be implemented so
		as to demonstrate a key operating system concept.  Kernel file systems such as 
		\texttt{procfs} or \texttt{sysfs} could also be implemented, if I have time.  Another
		workaround is to create some kind of \emph{RAM disk}.  It is preferred to implement
		a maybe-simplified version of contemporary file system, like Berkeley FFS, if time
		permitting.
	}
	
	The xv6 file system provides data files, which are uninterpreted byte arrays, and
	directories, which contain named references to data files and other directories. Xv6
	implements directories as a special kind of file. The directories form a tree, starting at
	a special directory called the \emph{root}. A path like \texttt{/a/b/c} refers to the file or directory
	named \texttt{c} inside the directory named \texttt{b} inside the directory named \texttt{a} in the root 
	directory \texttt{/}. Paths that don't begin with \texttt{/} are evaluated relative to the calling 
	process's current directory, which can be changed with the \texttt{chdir} system call. Both these code
	fragments open the same file (assuming all the directories involved exist):
	\begin{lstlisting}[style=c]
	chdir("/a");
	chdir("b");
	open("c", O_RDONLY);
	
	open("/a/b/c", O_RDONLY);
	\end{lstlisting}
	The first fragment changes the process's current directory to \texttt{/a/b}; the second neither
	refers to nor modifies the process's current directory.
	
	There are multiple system calls to create a new file or directory: \texttt{mkdir} creates a
	new directory, \texttt{open} with the \texttt{O\_CREATE} flag creates a new data file, and 
	\texttt{mknod} creates a new device file. This example illustrates all three:
	\begin{lstlisting}[style=c]
	mkdir("/dir");
	fd = open("/dir/file", O_CREATE|O_WRONLY);
	close(fd);
	mknod("/console", 1, 1);
	\end{lstlisting}
	\texttt{mknod} creates a file in the file system, but the file has no contents. Instead, the file's
	metadata marks it as a device file and records the major and minor device numbers
	(the two arguments to \texttt{mknod}), which uniquely identify a kernel device. When a
	process later opens the file, the kernel diverts \texttt{read} and \texttt{write} system calls to the kernel
	device implementation instead of passing them to the file system.
	
	\texttt{fstat} retrieves information about the object a file descriptor refers to. It fills in a
	\texttt{struct stat}, defined in \texttt{stat.h} as \textcolor{red}{
		\texttt{stat} structure may become different from original xv6 implementation.
	}:
	\begin{lstlisting}[style=c]
	#define T_DIR 1
	#define T_FILE 2
	#define T_DEV 3
	struct stat {
		short type;
		int dev;
		uint ino;
		short nlink;
		uint size;
	};
	\end{lstlisting}
	
	A file's name is distinct from the file itself; the same underlying file, called an 
	\texttt{inode}, can have multiple names, called \emph{links}. The \texttt{link} system call 
	creates another file
	system name referring to the same \texttt{inode} as an existing file. This fragment creates a
	new file named both \texttt{a} and \texttt{b}.
	\begin{lstlisting}[style=c]
	open("a", O_CREATE|O_WRONLY);
	link("a", "b");
	\end{lstlisting}
	Reading from or writing to \texttt{a} is the same as reading from or writing to \texttt{b}. Each inode
	is identified by a unique \emph{inode number}. After the code sequence above, it is possible to
	determine that a and b refer to the same underlying contents by inspecting the result
	of \texttt{fstat}: both will return the same inode number (\texttt{ino}), and the \texttt{nlink} count 
	will be
	set to 2.
	
	The \texttt{unlink} system call removes a name from the file system. The file's inode
	and the disk space holding its content are only freed when the file's link count is zero
	and no file descriptors refer to it. Thus adding
	\begin{lstlisting}[style=c]
	unlink("a");
	\end{lstlisting}
	to the last code sequence leaves the inode and file content accessible as b. Further-
	more,
	\begin{lstlisting}[style=c]
	fd = open("/tmp/xyz", O_CREATE|O_RDWR);
	unlink("/tmp/xyz");
	\end{lstlisting}
	is an idiomatic way to create a temporary inode that will be cleaned up when the 
	process closes \texttt{fd} or exits.
	
	Xv6 commands for file system operations are implemented as user-level programs
	such as \texttt{mkdir}, \texttt{ln}, \texttt{rm}, etc. This design allows anyone to 
	extend the shell with new user
	commands. In hind-sight this plan seems obvious, but other systems designed at the
	time of Unix often built such commands into the shell (and built the shell into the
	kernel).
	
	One exception is \texttt{cd}, which is built into the shell. \texttt{cd} must change the 
	current working directory of the shell itself. If \texttt{cd} were run as a regular command, then
	the shell would fork a child process, the child process would run \texttt{cd}, and \texttt{cd} would
	change the child's working directory. The parent's (i.e., the shell's) working directory
	would not change.
	
	\chapter{Operating system organization}
	A key requirement for an operating system is to support several activities. For
	example, using the system call interface described in chapter 0 a process can start new
	processes using \texttt{fork}. The operating system must arrange that these processes can
	\emph{time-share} the resources of the computer. For example, a process may start more
	new processes than there are processors in the computer, yet all processes must be able
	to make some progress. In addition, the operating system must arrange for \emph{isolation}
	between the processes. That is, if one process has a bug and fails, it shouldn't impact
	processes that don't have a dependency on the failed process. Complete isolation,
	however, is too strong, since it should be possible for processes to interact; for example,
	it is convenient for users to combine processes to perform complex tasks (e.g., by using
	pipes). Thus, the implementation of an operating system must achieve three require-
	ments: multiplexing, isolation, and interaction.
	
	This chapter provides an overview of how operating systems are organized to
	achieve these 3 requirements. It turns out there are many ways to do so, but this text
	focuses on mainstream designs centered around a \emph{monolithic kernel}, which is used
	by many Unix operating systems. This chapter illustrates this organization by tracing
	the first process that is created when xv6 starts running. In doing so, the text provides
	a glimpse of the implementation of all major abstractions that xv6 provides, how they
	interact, and how the three requirements of multiplexing, isolation, and interaction are
	met. Most of xv6 avoids special-casing the first process, and instead reuses code that
	xv6 must provide for standard operation. Subsequent chapters will explore each ab-
	straction in more detail.
	
	Xv6 runs on Intel 80386 or later ("x86") processors on a PC platform, and much
	of its low-level functionality (for example, its process implementation) is x86-specific.
	This book assumes the reader has done a bit of machine-level programming on some
	architecture, and will introduce x86-specific ideas as they come up. Appendix A briefly
	outlines the PC platform.
	
	\textcolor{red}{
		The original xv6 implementation mixes hardware-specific code and hardware-independent
		ones together.  xv6' primarily aims to port xv6 from x86 architecture onto Loongson 3A, a MIPS64
		CPU, in a so-called Loongson Multitech Board.  Xv6' also tries to decouple these two parts, 
		and potentially machine-dependent
		and architecture-dependent but machine-common codes.  
	}
	
	\section{Abstracting physical resources}
	The first question one might ask when encountering an operating system is why
	have it at all? That is, one could implement the system calls in \figurename{\ref{syscalls}} as a library,
	with which applications link. In this plan, each application could even have its own 
	library, perhaps tailored to its needs. In this plan, the application can directly interact
	with the hardware resources and use those resources in the best way for the application 
	(e.g., to achieve high performance or predictable performance). Some tiny operating 
	systems for embedded devices or real-time systems are organized in this way.
	
	The downside of this approach is that applications are free to use the library,
	which means they can also \emph{not} use it. If they don't use the operating system library,
	then the operating system cannot enforce time sharing. It must rely on the application
	to behave properly and, for example, periodically give up a processor so that another
	application can run. Such a \emph{cooperative} time-sharing scheme is maybe OK for a system 
	where all applications trust each other, but doesn't provide strong isolation if 
	applications are mutually distrustful.
	
	To achieve strong isolation a helpful approach is to disallow applications to have
	direct access to the hardware resources, but instead to abstract the resources into 
	services. For example, applications interact with a file system only through \texttt{open}, \texttt{read},
	\texttt{write}, and \texttt{close} system calls, instead of read and writing raw disk sectors. This 
	provides the application with the convenience of path names, and it allows the operating
	system (as the implementer of the interface) to manage the disk.
	
	Similarly, in Unix applications run as processes using \texttt{fork}, allowing the operating
	system to save and restore registers on behalf of the application when switching 
	between different processes, so that application don't have to be aware of process 
	switching. Furthermore, it allows the operating system to forcefully switch an application out
	of a processor, if the application, for example, is an end-less loop.
	
	As another example, Unix processes use \texttt{execve} to build up their memory image,
	instead of directly interacting with physical memory. This allows the operating system
	to decide where to place a process in memory and move things around if there is a
	shortage of memory, and provides applications with the convenience of a file system to
	store their images.
	
	To support controlled interaction between applications, Unix applications can use
	only file descriptors, instead of to make up some sharing convention of their own (e.g.,
	reserving a piece of physical memory). Unix file descriptors abstract all the sharing
	details away, hiding from the application if the interaction is happening with the 
	terminal, file system, or pipes, yet allows the operating system to control the interaction.
	For example, if one application fails, it can shut down the communication channel.
	
	As you can see, the system call interface in \figurename{\ref{syscalls}} is carefully designed to
	provide programmer convenience but also for the implementation of the interface to
	enforce strong isolation. The Unix interface is not the only way to abstract resources,
	but it has proven to be a very good one.
	
	\section{User mode, kernel mode, and system calls}
	To provide strong isolation between the software that uses system calls and the
	software that implements the system calls, we need a hard boundary between applications 
	and the operating system. If the application makes a mistake, we don't want the
	operating system to fail. Instead, the operating system should be able to clean up the
	application and continue running other applications. This strong isolation means that
	application shouldn't be able to write over data structures maintained by the operating
	system, shouldn't be able to overwrite instructions of the operating system, etc.
	
	To provide for such strong isolation processors provide hardware support. For
	example, the x86 processor, like many other processors, has two modes in which the
	processor executes instructions: \emph{kernel mode} and \emph{user mode}. In kernel mode the
	processor is allowed to execute \emph{privileged instructions}. For example, read and writing
	to the disk (or any other I/O device) is a privileged instruction. If an application
	in user mode attempts to execute a privileged instruction, then the processor doesn't
	execute the instruction, but switches to kernel mode so that the software in kernel
	mode can clean up the application, because it did something it shouldn't be doing.
	\textcolor{red}{
		MIPS processors have a \emph{supervisor mode} in addition to traditional
		user and kernel mode, but it is generally not used in modern operating system design.
	}
	Applications can execute only user mode instructions (e.g., adding numbers, etc.) and 
	is said to be running in \emph{user
	space}, while the software in kernel mode can execute also privileged instructions and
	is said to be running in \emph{kernel space}. The software running in kernel space (or in
	kernel mode) is called the \emph{kernel}.
	
	If a user-mode application must read or write to disk, it must transition to the
	kernel to do so, because the application itself can not execute I/O instructions. Processors 
	provide a special instruction that switches the processor from user mode to kernel
	mode and enters the kernel at an entry point specified by the kernel. (The x86 processor 
	provides the int instruction for this purpose, \textcolor{red}{
		whereas the MIPS processors propose various instructions like \texttt{syscall},
		\texttt{break}, or trap instructions, the \texttt{syscall} instruction mostly
		used.
	}) Once the processor has switched to
	kernel mode, the kernel can then validate the arguments of the system call, decide
	whether the application is allowed to perform the requested operation, and then deny
	it or execute it. It is important that the kernel sets the entry point when transition to
	kernel mode; if the application could decide the kernel entry point, a malicious 
	application could enter the kernel at a point where the validation of arguments etc. is
	skipped.
	
	\section{Kernel organization}
	A key design question for an operating system is what part of the operating system 
	should run in kernel mode. A simple answer is that the kernel interface is the system 
	call interface. That is, 
	\texttt{fork}, \texttt{exec}, \texttt{open}, \texttt{close}, \texttt{read}, \texttt{write}, 
	etc. are all kernel
	calls. This choice means that the complete implementation of the operating system
	runs in kernel mode. This kernel organization is called a \emph{monolithic kernel}.
	
	In this organization the complete operating system runs with full hardware privilege. 
	This organization is convenient because the OS designer doesn't have to decide
	which part of the operating system doesn't need full hardware privilege. Furthermore,
	it easy for different parts of the operating system to cooperate. For example, an operating 
	system might have a buffer cache that can be shared both by the file system and
	the virtual memory system.
	
	A downside of the monolithic organization is that the interfaces between different
	parts of the operating system are often complex (as we will see in the rest of this text),
	and therefore it is easy for an operating system developer to make a mistake. In a
	monolithic kernel, a mistake is fatal, because an error in kernel mode will often result
	in the kernel to fail. If the kernel fails, the computer stops working, and thus all 
	applications fail too. The computer must reboot to start again.
	
	To reduce the risk of mistakes in the kernel, OS designers can make the lines of
	code that run in kernel mode small. Most of the operating system doesn't need access
	to privileged instructions, and can thus run as ordinary user-level applications, with
	which applications interact with through messages. This kernel organization is called a
	\emph{microkernel}.
	
	In a microkernel, the kernel interface consists of a few low-level functions for
	starting applications, performing I/O, sending messages to applications, etc. This 
	organization allows the kernel to be implemented with a few lines of code, since it doesn't
	do much, as most functionality of the operating system is implemented by user-level
	servers.
	
	In the real-world, one can find both monolithic kernels and microkernels. For 
	example, Linux is mostly implemented as a monolithic kernel, although some OS 
	functions run as user-level servers (e.g., the windowing system). Xv6 is implemented as a
	monolithic kernel, following most Unix operating systems. Thus, in xv6, the kernel 
	interface corresponds to the operating system interface, and the kernel implements the
	complete operating system. Since xv6 doesn't provide many functions, its kernel is
	smaller than some microkernels.  \textcolor{red}{
		xv6' would inherit the monolithic kernel design from original xv6 implementation.
	}
	
	\section{Process overview}
	The unit of isolation in xv6 (as in other Unix operating systems) is a \emph{process}.
	The process abstraction prevents one process from wrecking or spying on another 
	process' memory, CPU, file descriptors, etc. It also prevents a process from wrecking the
	kernel itself (i.e., from preventing the kernel to enforce isolation). The kernel must
	implement the process abstraction with care because a buggy or malicious application
	may trick the kernel or hardware in doing something bad (e.g., circumventing enforced
	isolation). The mechanisms used by the kernel to implement processes include 
	user/kernel mode flag, address spaces, and time slicing of threads, which this subsection
	provides an overview of.
	
	To be able to enforce isolation, a process is an abstraction that provides the 
	illusion to a program that it has its own abstract machine. A process provides a program
	with what appears to be a private memory system, or address space, which other
	processes cannot read or write. A process also provides the program with what 
	appears to be its own CPU to execute the program's instructions.
	
	Xv6 uses page tables (which are implemented by hardware) to give each process
	its own address space. The x86 page table translates (or "maps") a \emph{virtual address}
	(the address that an x86 instruction manipulates) to a \emph{physical address} (an address
	that the processor chip sends to main memory).
	
	\begin{hilight}
		In xv6', since the memory management hardware merely consists of a TLB in MIPS
		architecture, most part of virtual-to-physical translation is done \emph{manually},
		that is, by software.  Memory management would be discussed in detail in chapter 3.
	\end{hilight}
	
	Xv6 maintains a separate page table for each process that defines that process's
	address space. An address space includes the process's
	user memory starting at virtual address zero. Instructions come first, followed by global 
	variables, then the stack, and finally a "heap" area (for malloc) that the process can
	expand as needed.
	
	Each process's address space maps the kernel's instructions and data as well as the
	user program's memory. When a process invokes a system call, the system call executes 
	in the kernel mappings of the process's address space. This arrangement exists
	so that the kernel's system call code can directly refer to user memory. In order to
	leave room for user memory to grow, xv6's address spaces map the kernel at high 
	addresses, starting at 0x80100000.  \textcolor{red}{
		xv6', running on MIPS, maps the kernel in a segment called \texttt{CKSEG0}
		\marginpar{\footnotesize\texttt{arch/mips/include/asm/\\
				addrspace.h:115}}, which
		corresponds to the lower 512MB of physical address.
	}
	
	The xv6 kernel maintains many pieces of state for each process, which it gathers
	into a \texttt{struct proc}. \textcolor{red}{
		xv6' treats process and threads alike as \emph{tasks}, a concept borrowed from
		Linux, and the state information is stored inside a
		\marginpar{\footnotesize
			\textcolor{red}{\texttt{include/sched/task.h:39}}}
		\texttt{task}
		structure.
	} A task's most important pieces of kernel state are its
	page table, its kernel stack, and its run state. We'll use the notation t->xxx 
	(or p->xxx for compatibility with original xv6) to refer to
	elements of the process/task structure.
	
	Each process has a thread of execution (or \emph{thread} for short) that executes the
	process's instructions. A thread can be suspended and later resumed. To switch
	transparently between processes, the kernel suspends the currently running thread and
	resumes another process's thread. Much of the state of a thread (local variables, function
	call return addresses) is stored on the thread's stacks. Each process has two stacks: a
	user stack and a kernel stack (\texttt{p->kstack}). When the process is executing user
	instructions, only its user stack is in use, and its kernel stack is empty. When the
	process enters the kernel (for a system call or interrupt), the kernel code executes on the
	process's kernel stack; while a process is in the kernel, its user stack still contains saved
	data, but isn't actively used. A process's thread alternates between actively using its
	user stack and its kernel stack. The kernel stack is separate (and protected from user
	code) so that the kernel can execute even if a process has wrecked its user stack.
	
	\begin{hilight}
		If time permitting, xv6' would support processes with multiple threads.  This is not
		my major goal since threads and processes don't differ very much in design of Linux, and thus,
		xv6'.
	\end{hilight}
	
	When a process makes a system call, the processor switches to the kernel stack,
	raises the hardware privilege level, and starts executing the kernel instructions that 
	implement the system call. When the system call completes, the kernel returns to user
	space: the hardware lowers its privilege level, switches back to the user stack, and 
	resumes executing user instructions just after the system call instruction. A process's
	thread can "block" in the kernel to wait for I/O, and resume where it left off when the
	I/O has finished.
	
	\textcolor{red}{In original xv6}, \texttt{p->state} indicates whether the process is 
	allocated, ready to run, running, waiting for I/O, or exiting.  \textcolor{red}{
		This is slightly different in xv6', where state of a process is indicated by
		\texttt{t->state} and \texttt{t->flags} together.
	}
	
	\textcolor{red}{In the old implementation} \texttt{p->pgdir} holds the process's page table, 
	in the format that the x86 hardware expects. xv6 causes the paging hardware to use a 
	process's \texttt{p->pgdir} when executing
	that process. A process's page table also serves as the record of the addresses of the
	physical pages allocated to store the process's memory.  \textcolor{red}{
		In xv6' the page table is stored inside \texttt{t->mm}, a pointer pointing to a memory 
		mapping structure, which is split into two parts, one hardware-dependent and the other
		hardware-independent.  The hardware-related part is stored in an inner structure
		\marginpar{
			\footnotesize\ttfamily
			include/mm/vmm.h:86
		} \texttt{arch\_mm\_t}.  In current implementation,
		\marginpar{
			\footnotesize\ttfamily
			arch/mips/include/\\
			asm/mm/hier/vmm.h:20
		}
		\texttt{arch\_mm\_t} merely holds the address of page global directory of the process.
		This structure organization is equivalent to that of original implementation, with the
		advantage of decoupling hardware-specific page table and hardware-irrelevant memory
		management code.
	}
	
	\section{Code: the first address space}
	To make the xv6 organization more concrete, we look how the kernel creates the first
	address space (for itself), how the kernel creates and starts the first process, and the
	first system call that the first process makes. By tracing these operations we see in detail
	how xv6 provides strong isolation for processes. The first step in providing strong 
	isolation is setting up the kernel to run in its own address space.
	
	\begin{hilight}
		When it comes to Loongson Multitech Board, the machine bootstraps
		itself by loading program from a built-in PMON ROM which is already provided by the
		manufacturer.  PMON downloads the operating system kernel remotely via TFTP after hardware
		initialization, and automatically detects the entry point and jumps there.  The kernel must
		reside in \texttt{CKSEG0}, a segment inaccessible from other modes, 
		\marginpar{
			\footnotesize\ttfamily arch/mips/include/\\
			asm/addrspace.h:115
		}
		where virtual addresses are linearly mapped to physical address by taking
		the lower 29 bits, not passing through TLB during translation.  By convention the
		lower 256MB to 512MB actually corresponds to I/O devices or ROM, so the kernel can only
		take the lower 0-256MB address.  The phenomenon one can observe is that xv6' loads itself
		in virtual address \texttt{0xffff ffff 8030 0000}.
		
		In MIPS64 architecture, the virtual address layout is fixed, and is described
		in \texttt{arch/mips/include/asm/addrspace.h}.  By enforcing the virtual addresses
		directly used by kernel to reside in \texttt{XKPHY}, a kernel-mode-only address space,
		\marginpar{
			\footnotesize\ttfamily
			arch/mips/include\\
			asm/addrspace.h:92
		}
		where virtual addresses are mapped to physical ones in a manner similar to that in \texttt{CKSEG0},
		it is no longer necessary to make separate virtual address mapping for kernel, and to switch page
		table during user/kernel mode transition.  However, the address space \texttt{XKPHY} is
		MIPS64-specific.  In 32-bit MIPS architecture a different mechanism should be developed.

		The kernel entry points to a small piece of assembly code,
		\marginpar{
			\footnotesize\ttfamily
			arch/mips/entry.S
		}
		which is almost always architecture-specific.  In xv6', all the assembly snippet would do is
		initializing various co-processor registers, boot arguments, initial kernel stack and processor
		information, and then transfer control to C code.
		\marginpar{
			\footnotesize\ttfamily
			\textcolor{red}{kern/init.c}
		}
	\end{hilight}
	
	\section{Code: creating the first process}
	Now the kernel runs within its own address space, we look at how the kernel creates 
	user-level processes and ensures strong isolation between the kernel and user-level
	processes, and between processes themselves.
	
	\begin{hilight}
		The procedure which spawns initial tasks is \texttt{task\_init}.
		\marginpar{
			\footnotesize\ttfamily
			kern/sched/task.c:182
		}  \texttt{task\_init} spawns two tasks named \texttt{idle} (\texttt{idle\_init})
		\marginpar{
			\footnotesize\ttfamily
			kern/sched/task.c:140
		}
		and
		\texttt{init} (\texttt{initproc\_init}).
		\marginpar{
			\footnotesize\ttfamily
			kern/sched/task.c:88
		}
		\texttt{initproc\_init} dynamically allocates a \texttt{task} structure first by
		calling \texttt{task\_new},
		\marginpar{
			\footnotesize\ttfamily
			kern/sched/task.c:26
		}which is called on allocating every new task.  The initializer then allocates and 
		sets up a new memory mapping object, creating a new page table on the way
		\marginpar{
			\footnotesize\ttfamily
			\textcolor{red}{kern/sched/task.c:48}
		}
		(\texttt{task\_setup\_mm}).

		The initializer should then set up the process context and the kernel stack.
		\texttt{task\_setup\_kstack}
		\marginpar{
			\footnotesize\ttfamily
			kern/sched/task.c:26
		} tries to allocate a kernel stack in kernel space first, and then preserves
		a portion on the top for storing process context for context switches.
		The procedure for setting up initial process context is architecture-specific as
		shown in \texttt{task\_bootstrap\_context}
		\marginpar{
			\footnotesize\ttfamily
			arch/mips/sched/\\task.c:41
		}.  In short, this routine would make the process return to the entry of C
		function \texttt{forkret}
		\marginpar{
			\footnotesize\ttfamily
			kern/syscall/fork.c:26
		}, which would in turn transfer control to function \texttt{arch\_forkret}
		\marginpar{
			\footnotesize\ttfamily
			arch/mips/syscall/\\forkret.S
		} written in assembly, which would restore a previously built trap frame
		for restoring and returning in a way similar to that of finishing handling
		an exception.  This setup is mostly the same for ordinary \texttt{fork}s.

		Now the initializer should build a trap frame inside the kernel stack in a
		manner so as to make the process have the illusion that it is returning from
		a trap handler after a context switch.  Building such trap frame is
		architecture-dependent, and is accomplished by
		\begin{enumerate}
			\item \texttt{task\_init\_trapframe}
			\marginpar{
				\footnotesize\ttfamily
				arch/mips/sched/\\task.c:21
			} (for filling some process-common trap frame entries),
			\item \texttt{set\_task\_user}
			\marginpar{
				\footnotesize\ttfamily
				arch/mips/sched/\\task.c:53
			} (for setting the task to run in user mode),
			\item \texttt{set\_task\_enable\_intr}
			\marginpar{
				\footnotesize\ttfamily
				arch/mips/sched/\\task.c:62
			} (to enable interrupt),
			\item \texttt{set\_task\_startsp}
			\marginpar{
				\footnotesize\ttfamily
				arch/mips/sched/\\task.c:68
			} (for setting up stack pointer register to point to top of user stack,
			called by \texttt{set\_task\_ustack}
			\marginpar{
				\footnotesize\ttfamily
				kern/sched/task.c:71
			}
			for user stack setup),
			\item \texttt{set\_task\_main\_args}
			\marginpar{
				\footnotesize\ttfamily
				arch/mips/sched/\\task.c:107
			} (for passing arguments of \texttt{main()}), and
			\item \texttt{set\_task\_entry}
			\marginpar{
				\footnotesize\ttfamily
				arch/mips/sched/\\task.c:115
			} (for specifying program entry) together.
		\end{enumerate}
		After execution of \texttt{set\_task\_main\_args},
		arguments should be placed properly above the per-process user stack, which should reside in
		a separate user page.  User stack allocation is done by \texttt{set\_task\_ustack}
		\marginpar{
			\footnotesize\ttfamily
			\textcolor{red}{kern/sched/task.c:71}
		}

		Unlike original xv6, the first process is loaded from a separate ELF program, named \texttt{init},
		preferably stored on the disk, but now embedded in the kernel binary file as the box currently lacks
		external storage.
		\texttt{init} would become the xv6' equivalent of contemporary \texttt{init}
		program in System V or BSD, or \texttt{systemd} in Linux, but currently it does nothing rather than
		emitting a bunch of lines to the serial.
		\marginpar{
			\footnotesize\ttfamily
			\textcolor{red}{ramdisk/init/init.c}
		}
		To actually load the program into memory and execute it, the kernel locates the embedded
		\texttt{init} binary file by referencing the symbol \texttt{\_binary\_ramdisk\_init\_init\_start},
		\marginpar{
			\footnotesize\ttfamily
			\textcolor{red}{kern/sched/task.c:113}
		}
		generated by GNU \texttt{ld} utility while linking.  The procedure \texttt{task\_load\_elf\_kmem}
		\marginpar{
			\footnotesize\ttfamily
			\textcolor{red}{kern/syscall/\\execve.c:118}
		}
		loads each loadable segment into memory, mapping them to virtual address specified by the ELF
		program headers.  The creation is complete after the entry is located by looking into the
		corresponding field inside the ELF header.

		The process is now ready to run, and the initializer finalizes the procedure by setting appropriate
		PID, process name, and state (\texttt{TASK\_RUNNABLE})
	\end{hilight}
	
	\section{Code: running the first process}
	Now that the first process's state is prepared, it is time to run it. After \texttt{main} bootstrapped
	everything, \texttt{mpmain} calls \texttt{sched}
	\marginpar{
		\footnotesize\ttfamily
		\hilite{kern/sched/sched.c}
	} to start running processes (\hilite{NYI}). 
	\texttt{sched} (\hilite{NYI})
	looks for a process with \texttt{p->state} set to \texttt{TASK\_RUNNABLE}, and there's only one:
	\texttt{init}. It
	sets the per-cpu variable \texttt{current\_task}
	\marginpar{
		\footnotesize\ttfamily
		\hilite{arch/mips/include/\\asm/thread\_info.h:47}
	}
	to the process it found and \hilite{
		switches page table during context switch in \texttt{switch\_context},
		\marginpar{
			\footnotesize\ttfamily
			\hilite{arch/mips/sched/switch.S}
		} by exploiting and changing the ASID field in \texttt{CP0\_ENTRYHI}
		register
	}. \hilite{In original xv6 implementation on x86,} switchuvm also sets up a task
	state segment \texttt{SEG\_TSS} that instructs the hardware to execute system calls and 
	interrupts on the process's kernel stack. \hilite{
		This is no longer required on MIPS hardware, since user/kernel stack switching is done
		\emph{manually}.  User/kernel stack switching, along with other elements of exception
		handling, will be explored in chapter 4.
	}
	
	\texttt{sched} now sets p->state to RUNNING and calls \texttt{switch\_context} to perform a
	context switch to the target process's kernel thread. \texttt{switch\_context} saves the current registers
	and loads the saved registers of the target kernel thread (proc->context) into the
	hardware registers. \hilite{
		Unlike original xv6, \texttt{switch\_context} is implemented like an ordinary function.
		So the current context is \emph{inside} the process's context switch,
		since \texttt{switch\_context} is invoked by \texttt{sched} function, which is periodically called
		during clock interrupts.  As the current context is saved \emph{during} a switch, the processor
		would run in the switching procedure of another process after restoring context registers.
		\texttt{switch\_context} also involves properly changing the current kernel stack.
	} We'll examine \texttt{switch\_context} in more detail in Chapter 5.
	
	\begin{hilight}
		Consider the very initial execution of \texttt{init}. As \texttt{task\_bootstrap\_context}
		\marginpar{
			\footnotesize\ttfamily
			arch/mips/sched/\\task.c:41
		} sets the \texttt{ra} register, i.e. the returning address to the entry of \texttt{forkret},
		with \texttt{a0} register pointing to the trap frame structure built by \texttt{initproc\_init}.
		After returning from \texttt{switch\_context}, the processor found itself at the starting point
		of \texttt{forkret},
		\marginpar{
			\footnotesize\ttfamily
			kern/syscall/fork.c:26
		}
		with the trap frame as argument.  As \texttt{forkret} is executed only once
		per process (whether by \texttt{fork()} or by creating \texttt{init}), it could be designed as a
		\emph{no-return} function.  \texttt{forkret} passes the received trap frame to assembly code
		\texttt{arch\_forkret}, which is also a no-return function.  \texttt{arch\_forkret}
		\marginpar{
			\footnotesize\ttfamily
			arch/mips/syscall/\\forkret.S
		} restores the trap frame in the same way as returning from an exception.
	\end{hilight}
	
	At this point, \hilite{
		\texttt{sp} and program counter (set by \texttt{eret} instruction
		according to \texttt{CP0\_EPC} register) are set to appropriate values.
	} These are virtual addresses
	in the process's address space. The processor's paging hardware translates them into
	physical addresses. The fact that \hilite{
		\texttt{set\_task\_user}
		\marginpar{
			\footnotesize\ttfamily
			arch/mips/sched/\\task.c:53
		} disables CP0 and sets the processor to work in user mode
	}
	means that the user code can only use pages \hilite{
		within \texttt{XUSEG}
		\marginpar{
			\footnotesize\ttfamily
			\hilite{
				arch/mips/include\\
				asm/addrspace.h:44
			}
		}
	},
	and cannot modify sensitive hardware registers such as \hilite{CP0}.
	So the process is constrained to using only its own memory.
	
	\section{The first system call: \hilite{\texttt{write}}}
	Now we have seen how the kernel provides strong isolation for processes, let's see
	how a user-level process can enter back into the kernel to ask for services that it
	cannot perform itself.
	
	\begin{hilight}
		As \texttt{execve} system call depends on a file system, which is not implemented
		yet, we'll consider \texttt{write} system call here.  But even \texttt{write} can
		do nothing other than output a bunch of characters into serial console at present.

		\texttt{write} is the first system call invoked by the user process \texttt{init}.
		It requests kernel service according to a simplified version of Linux/MIPS64 system
		call standard:
		\begin{enumerate}
			\item System calls are always requested by executing \texttt{syscall} instruction.
			\item System call number is put in register \texttt{v0(\$2)}.
			\item Arguments are put in \texttt{a0(\$4)} to \texttt{a7(\$11)} registers.
			\marginpar{
				\footnotesize
				\texttt{n64} \texttt{a4} to \texttt{a7} registers are named \texttt{t0} to
				\texttt{t3} in current code, while \texttt{n64} \texttt{t0} to \texttt{t3}
				are named \texttt{t4} to \texttt{t7} due to \texttt{o32} legacy ABI naming
				in current code.  Someday we'll fix that.
			}
			A C library is preferred to wrap user land system call functions into kernel-flavored
			argument format in case of dealing with more-than-eight arguments.  But since few system
			calls have so many parameters (there's no such system call in OpenBSD after all),
			it is fine to simply ignore the case, and just put everything in the registers.
			When it comes to \texttt{o32} ABI for MIPS32, things may become more complex.
			\item Return values are placed in \texttt{v0(\$2)}, or both \texttt{v0(\$2)} and
			\texttt{v1(\$3)} when one returns more values, e.g. returning from \texttt{pipe}.
			\item Error number is stored in register \texttt{a3(\$7)}.
		\end{enumerate}
		
		When \texttt{init} execute the following C code:
		\marginpar{
			\footnotesize\ttfamily
			ramdisk/init/init.c:21
		}
		\begin{lstlisting}[style=c]
		printf("%s\r\n", hellomsg);
		\end{lstlisting}
		The C library transforms the print request into something like the following system call:
		\begin{lstlisting}[style=c]
		write(STDERR_FILENO, buf, strlen(buf));
		\end{lstlisting}
		As the arguments are automatically put into the argument registers by C compiler following
		the \texttt{n64} ABI, the \texttt{write}
		\marginpar{
			\footnotesize\ttfamily
			lib/libc/arch/\\mips/syscall.S
		}
		function only needs to set up \texttt{v0}, and store
		error number register into a global symbol \texttt{errno} (NYI), since the returning value is already
		stored in the return value register \texttt{v0}.
		
		After the processor catches a system call exception,
		\marginpar{
			\footnotesize\ttfamily
			arch/mips/traps.c:442
		}
		it looks up a system call table according
		to number placed in \texttt{v0}, jumps into the corresponding handler, and begin servicing the
		user process.
		\marginpar{
			\footnotesize
			\texttt{arch/mips/syscall/\\syscall.c:23},
		}In \texttt{write}'s case, the service handler is \texttt{do\_write}.
		\marginpar{
			\footnotesize\ttfamily
			\hilite{fs/write.c}
		}
	\end{hilight}
	
	\chapter{Page tables}
	Page tables are the mechanism through which the operating system controls what
	memory addresses mean. They allow xv6 to multiplex the address spaces of different
	processes onto a single physical memory, and to protect the memories of different
	processes. The level of indirection provided by page tables is also a source for many neat
	tricks. xv6 uses page tables primarily to multiplex address spaces and to protect
	memory. It also uses a few simple page-table tricks: mapping the same memory (the
	kernel) in several address spaces, mapping the same memory more than once in one
	address space (each user page is also mapped into the kernel's physical view of memory),
	and guarding a user stack with an unmapped page. The rest of this chapter explains
	the page tables that the \hilite{MIPS64} hardware provides and how xv6 uses them.
	
	\section{Paging hardware}
	\begin{hilight}
		Literally, paging hardware deal with \emph{pages}, an aligned memory chunk of fixed size.
		Different from x86, MIPS paging hardware \textbf{only consists of a TLB}.
		The TLB have a number of entries (in case of Loongson 3A, 64 entries), each
		containing a pair of adjacent virtual page numbers (VPN), and their mapping to physical
		frame numbers (PFN), as well as permission flags.
		
		When a virtual address is requested, either for reading or for writing, TLB
		tries to look up the virtual page number in its entry table.  The hardware throws an
		exception if:
		\begin{enumerate}
			\item the VPN does not exist in TLB (\emph{TLB loading miss} or \emph{TLB storing miss},
			depending on operation), or
			\item the VPN does exist but is marked invalid, due to, for example, being swapped out
			to external storage (\emph{TLB invalid load} or \emph{TLB invalid store}), or
			\item the process tries to write on a read-only virtual page (\emph{TLB modify}).
		\end{enumerate}
		TLB then leave everything to the operating system to handle these exception.
		
		Side note: MIPS does not provide execution prevention similar to NX bit on \texttt{amd64} platform.
		Loongson 2F used to have a similar feature, which is removed in Loongson 3A.
	\end{hilight}
	
	\section{\hilite{Cache aliasing -- a MIPS specific issue}}
	\begin{hilight}
		For speeding up the processor, MIPS radically indexes its primary cache by virtual address.
		This leads into a MIPS-specific problem called \emph{cache aliasing}.  The symptom is that
		a physical address may appear on multiple cache lines, each with possibly different data.
		Secondary cache is free from this issue.
		
		In Loongson 3A, the primary data cache is indexed by the last 14 bit of virtual address,
		with the remaining high 24 bits serving as cache tag.  If the page size is 4KB, i.e. VPN
		is the high 26 bits of a virtual address, the last 2 bits of VPN would become part of the
		cache index.  If two virtual pages with VPNs different in the last 2 bits happen to point
		to the same physical page, it is possible to have different cache lines in a same cache set,
		referencing the same physical address, but caching different data, breaking coherency.
		
		A simple workaround is to tune the page size so as to prevent any bits from VPN being used as
		cache index.  In Loongson 3A, a page size of 16KB is a good choice.
	\end{hilight}
	
	\section{\hilite{Page table organization}}
	\begin{hilight}
		Since MIPS does not have hardware other than TLB to support paging, the operating system may
		maintain a page table by its choice.  xv6' utilizes a page table hierarchy similar to the
		classic two-level hierarchy on 32-bit x86 processors.
		
		Due to previously discussed \emph{cache aliasing} issue, the page size on Loongson 3A is chosen
		to be 16KB.  That is, the higher 24 bits are VPN, while the lower 14 bits represent page offset.
		
		A plain attempt of maintaining a page table is to keep a huge array of pointers
		pointing to virtual pages.  This approach is quite implausible and inefficient since such kind
		of table needs $2^{24}\text{ pointers}\times 8\text{ bytes per pointer}=128$MB \textbf{per process},
		storing nothing but pointers, while leaving most entries unused since most programs won't need that
		many pages.
		
		One can employ a hierarchical scheme for reducing memory consumption of page tables.  That is, to
		organize the page table as a page table \emph{tree}.
		We can define \emph{leaf page tables}, pointing
		to virtual pages to be used by user programs, and \emph{internal page tables}, which points to pages
		storing another level of page tables, internal or leaf.  This way, a branch containing no
		user pages could be easily pruned by maintaining page entry count inside the page structure,
		recycling zero-count page tables during page freeing.
		\marginpar{
			\footnotesize\ttfamily
			\hilite{arch/mips/mm/hier/\\pgtable.c:199}
		}
		
		The leaf and internal page tables themselves
		are stored in a one-table-per-page manner.  Therefore, a one-page-sized page table could point to
		$2^{14}/8=2048$ pages
		when page size is 16KB.  The 11 bits next to the lowest 14 bits could be used as index of the leaf
		page tables, another higher 11 bits as index of the \emph{page middle directories} (PMD), the last
		11 bits as that of the \emph{page global directory} (PGD).  The highest bit of virtual address is
		discarded and is always zero.
		
		TLB refilling for dealing with TLB loading/storing misses becomes quite straightforward:
		
		Hierarchical page tables share similar structures and the same rationale across different
		architectures.  The \texttt{amd64}
		architecture uses a 4-level page table tree to deal with 4KB-paged 48-bit virtual addresses, each page
		table pointing to 512 next-level pages.
	\end{hilight}
	
	\section{\hilite{MIPS Physical address convention}}
	\begin{hilight}
		It is previously mentioned that virtual address space \texttt{CKSEG0} corresponds to the lower
		512MB physical address, while \texttt{XKPHY} can cover as large as $2^48$ bytes space.  However,
		a further low-level mapping exist between physical addresses and \emph{real hardware}.
		
		\begin{enumerate}
			\item Physical address \texttt{0x0} to \texttt{0x0fff ffff} maps to lower
			256MB of main memory.
			\item Usually, physical address \texttt{0x1000 0000} to \texttt{0x1fff ffff} corresponds to
			I/O devices, BIOS (starting from \texttt{0x1fc0 0000}), and other hardware.  These addresses
			are normally directly accessed, not passing through caches during the process.
			\item Method of accessing memory higher than 256MB generally varies between machines.
			In case of Loongson 3A, the physical addresses mapped to memory addresses higher than
			256MB equals to the memory address itself plus the overall memory size.  For example, if the
			total memory capacity is 4GB, the physical address range corresponding to memory higher than
			256MB is \texttt{0x0001 1000 0000} to \texttt{0x0001 ffff ffff}.  \texttt{setup\_page\_array}
			\marginpar{
				\footnotesize\ttfamily
				\hilite{arch/mips/mach/\\loongson3a5/mm.c:14}
			}
			installs the array of \texttt{struct page} at high memory (see section 2.7), just above the lower 256MB,
			calculating the high physical memory offset during initialization.
		\end{enumerate}
	\end{hilight}
	
	\section{Process address space}
	Each process has a separate page table, and xv6 tells the page table hardware to
	switch page tables when xv6 switches between processes. A
	process's user memory resides in \hilite{\texttt{XUSEG}}, starts at virtual address zero 
	and can grow up to \hilite{\texttt{XSSEG}},
	allowing a process to address up to \hilite{256 TB} of memory (if there's really that much physical memory).
	The file \hilite{\texttt{arch/mips/include/asm/addrspace.h}}
	\marginpar{
		\footnotesize\ttfamily
		\hilite{arch/mips/include/\\asm/addrspace.h}
	}
	declares the constants for xv6's memory layout, and macros to convert virtual to physical
	addresses \hilite{are defined in \texttt{include/mm/mmap.h}}.
	\marginpar{
		\footnotesize\ttfamily
		\hilite{include/mm/mmap.h}
	}
	
	When a process asks xv6 for more memory, xv6 first finds free physical pages to
	provide the storage, and then adds PTEs to the process's page table that point to the
	new physical pages. \hilite{xv6' sets the PTE\_DIRTY and PTE\_VALID flags in these PTEs when it's
	running on MIPS processors}. Most
	processes do not use the entire user address space; xv6' leaves \hilite{the entire PTE
	clear if it's not used}. Different processes' page tables translate user addresses to different pages of
	physical memory, so that each process has private user memory.
	
	Xv6' includes all mappings needed for the kernel to run in every process's page ta-
	ble; these mappings \hilite{
		are automatically enforced by MIPS hardware.  The kernel's instruction and static data
		resides in \texttt{CKSEG0}, whereas the kernel's dynamically allocated data are in
		\texttt{XKPHY}, both address spaces directly mapped to physical pages without looking up
		in TLB, both protected by hardware so that they are only accessible in kernel mode.  This design
		enables the kernel to read from and write to arbitrary positions, and negates the need
		of switching page tables repeatedly during user/kernel mode transition since in this way the
		kernel itself does \emph{not} require a page table to run.
	}
	
	To review, xv6 ensures that each process can only use its own memory, and that
	each process sees its memory as having contiguous virtual addresses starting at zero.
	xv6' implements the first \hilite{by only filling valid information in} PTEs of virtual addresses
	that refer to the process's own memory. It implements the second using the ability of
	page tables to translate successive virtual addresses to whatever physical pages happen
	to be allocated to the process.
	
	\section{Code: creating an address space}
	\begin{hilight}
		The original xv6 implementation employs a \texttt{setupkvm} routine for initializing the
		kernel page mapping.  This is no longer needed in xv6'.
		
		For user virtual memory, or other places where translation via a page table is needed,
		one can install a mapping from a range of virtual addresses to a list of physical ones
		by calling \texttt{map\_pages}.
		\marginpar{
			\footnotesize\ttfamily
			\hilite{kern/mm/vm.c:126}
		}  \texttt{map\_pages} maintains a list of virtual memory areas, and then establish
		mapping page by page by inserting PTEs for each virtual page via calling \texttt{arch\_map\_page}.
		\texttt{map\_pages} can also serve as basis of the \texttt{mmap} system call.
		
		\texttt{arch\_map\_page}
		\marginpar{
			\footnotesize\ttfamily
			\hilite{arch/mips/mm/hier/\\pgtable.c:238}
		} installs one page into the page table, connecting the VPN to PFN, by calling \texttt{pgtable\_insert}.
		\marginpar{
			\footnotesize\ttfamily
			\hilite{arch/mips/mm/hier/\\pgtable.c:154}
		}
		It is not strictly hardware-specific since page tables on different hardware may look the same.
		The reason I put page-table-related code inside the \texttt{arch} directory is that one
		can potentially implement any page table structure he prefer, since MIPS does not have
		paging hardware other than TLB. \texttt{pgtable\_insert} traverses the page table hierarchy and
		fill the PTE position located by \texttt{pgtable\_get}
		\marginpar{
			\footnotesize\ttfamily
			\hilite{arch/mips/mm/hier/\\pgtable.c:88}
		} with appropriate value.
	\end{hilight}
	
	\section{\hilite{Per-page property maintenance}}
	\marginpar{
		\footnotesize\ttfamily
		\hilite{
			kern/mm/init.c\\
			include/mm/mmap.h
		}
	}
	\begin{hilight}
		The maintenance of various state information of pages needs memory, which in turn needs some of the
		pages.  xv6' partitioned a portion of memory for specifically storing an array of page structures
		(\texttt{struct page})
		corresponding to all memory higher than 256MB.  The page structures related to pages storing this page
		array partition are marked explicitly as for page structure array, and are reserved throughout the execution
		of the operating system, and thus, never participate in dynamic page allocation, protecting the page array
		from being corrupted by other kernel activities.
	\end{hilight}
	
	\section{Physical memory allocation}
	The kernel needs to allocate and free physical memory at run-time for page tables, 
	process user memory, kernel stacks, and pipe buffers.
	
	\begin{hilight}
		\marginpar{
			\footnotesize\ttfamily
			\hilite{
				kern/mm/slab.c\\
				include/mm/slab.h
			}
		}
		The original xv6 implementation directly always allocates memory in full pages no matter what size the kernel
		actually requested.  This is highly inefficient, especially when page size is larger, since occupying a
		16KB-sized chunk to maintain a 64-byte data structure is almost certainly not a good idea.  A simplified
		slab allocation, borrowed from Linux's SLUB implementation, is thus introduced to address this issue.
		
		The core of \emph{slab allocation} is to divide a page or several physically-continuous pages into smaller, 
		equally-sized chunks; when the requested size of allocation matches the size of chunk, the allocator returns
		one of the available chunk as a result.  The page or continuous pages storing those chunks are called a
		\emph{slab}, while the set of \emph{slabs} storing one particular size of chunks is called a
		\emph{slab cache}.  Each slab cache corresponds to one and only one specific chunk size.  In xv6', a slab
		always consists of one page.
		
		Theoretically, strict slab allocation leaves \emph{no} fragments as long as each size have one
		corresponding slab cache, as long as the probabilities of different requested sizes are equal.
		However, consider the following case:
		\begin{lstlisting}[style=c]
		/* The page size is 16KB */
		for (i = 520; i <= 1024; i += 8) {
			kmalloc(i);
		}
		\end{lstlisting}
		If the operating system is to prepare a slab cache dedicated to each requested size, a lot of
		poorly-utilized slabs, each of which only holds one object, would be created.  It is somehow more
		preferable to gather these sizes into one larger cache.  In xv6', the slab allocator defines 13
		caches, each to hold a range of object sizes.
		
		When slab allocator receive a request, it first:
		\begin{enumerate}
			\item find whether there is an available chunk inside a slab in a proper cache.  Return the chunk
			if such chunk is found, or
			\item allocate a page and mark it as a slab, and add it into the list of slabs in the cache.
		\end{enumerate}
		
		Slab allocation in xv6' deals with requests not greater than half the page size, aligned to 8 bytes.
		If requests are greater than half of page size, the operating system reverts to allocate and
		free whole pages.
		
		The operating system should maintain a list of available chunks per slab.  xv6' reuses the chunks
		which are not yet allocated to keep track of the index of next available chunk inside the slab,
		as it is quite costly to allocate such list elsewhere.
	\end{hilight}
\end{document}